\documentclass{article}
\usepackage{graphicx}
\usepackage{amsmath}
\usepackage{enumitem}
\usepackage{float}
\usepackage{listings}
\usepackage{xcolor}
\usepackage[a4paper, margin=1in]{geometry}

% Custom information
\newcommand{\className}{Course: Automatic Control Systems – ASEN 5114-001 – Spring 2025}
\newcommand{\professorName}{Professor: Dale Lawrence}
\newcommand{\taName}{Teaching Assistant: Anantha Dhruva}
\title{Homework 2 \\ \className \\ \professorName \\ \taName}
\author{Steve Gillet}
\date{\today}

\lstdefinestyle{matlabstyle}{
    language=Matlab,              % Specify the language
    basicstyle=\ttfamily\footnotesize\color{black}, % Code font
    keywordstyle=\color{blue}\bfseries, % Keywords in blue
    stringstyle=\color{green},    % Strings in green
    commentstyle=\color{magenta}, % Comments in magenta
    numbers=left,                 % Line numbers on the left
    numberstyle=\tiny\color{black},% Line number style
    stepnumber=1,                 % Line number increment
    breaklines=true,              % Line breaking
    frame=single,                 % Border around code
    backgroundcolor=\color{white},
    tabsize=4,                    % Tab size
    showstringspaces=false,       % Don't show spaces in strings
}

\begin{document}

\maketitle
\textit{
    "Use the plant and controller models from Homework 1 to explore the effect of control gains on
    closed loop system poles, and hence the effect on the closed loop natural response of the load
    shaft angle."
}

\section*{1.}

\textit{
    "Find the poles of the plant from your transfer function model from Experiment 1. Plot these
    in the complex plane."
}

\section*{2.}

\textit{
    "Find the poles of the closed loop system from part 5 of Homework 1. Plot these on the same
    plot as in part 1. above."
}

\section*{3.}

\textit{
    "Repeat part 2. above, using a gain factor g in the control law that varies between 0 and 1, in
    100 steps. Plot this 'root locus with respect to g' on the same plot as above. Describe the
    effect of this control system gain on the closed loop poles."
}

\section*{4.}

\textit{
    "Repeat part 3. above but use gain values g between 1 and 100. Describe the effect on the
    closed loop poles."
}

\section*{5.}

\textit{
    "Repeat part 4. but use negative g values between 0 and -1."
}

\section*{6.}

\textit{
    "What does the above root locus analysis tell you about the effect of the gain g on the closed
    loop system’s natural response? Verify your conclusions by simulating the closed loop step
    response for several values for g."
}

\end{document}